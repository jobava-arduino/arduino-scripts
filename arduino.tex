\documentclass[blue]{beamer}
\usepackage{beamerthemesplit}
\usepackage{amsmath,graphicx,amssymb,amsfonts,pgfarrows,pgfnodes,amssymb,amsfonts}
\usepackage{hyperref}
\usepackage{graphicx}
\hypersetup{pdfpagemode=FullScreen} 
\usetheme{Warsaw}
\useinnertheme{circles}
\useinnertheme{rounded}
\useoutertheme[subsection=false]{smoothbars}
\usecolortheme{sidebartab}


\title{Soyuz 11}
\author{Thomas Swartz}
\institute{The University of Scranton}
\date{\today}


\begin{document}

\frame{\titlepage}


\section{The Mission}

\begin{frame}
\frametitle{Mission Impossible?}
Salyut 1 posed an interesting problem for the Soviet Space Program; although they had a space station in orbit of the Earth, the process of transferring a crew to and from the station was never performed. \\
In April 25th, Soyuz 10 was sent to dock with the station but was unable to perfectly match with the dock, thus aborting the mission.\\ 
Soyuz 11 featured a redesigned docking mechanism to make it easier.
\end{frame}

\begin{frame}
\frametitle{Mission Impossible?}
\begin{center}
	\includegraphics[width=0.50\textwidth]{salyut.png}
\end{center}
\end{frame}

\section{So What Happened?}

\begin{frame}
\begin{center}\huge So Why Did the Valve Fail?
\end{center}
\end{frame}

\section{How Has it Changed Space Programs?}
\begin{frame}
\frametitle{How Has it Changed Space Programs?}
\begin{itemize}
\item<1-> The explosive bolts are sparsely used
\item<2-> The astronauts must wear protective clothing
\item<3-> All astronauts wear biological sensors
\item<4-> Radio contact at all points of re-entry
\item<5-> Automatic pressure regulation with backup systems
\end{itemize}
\end{frame}

\begin{frame}
\begin{center}\huge{Questions?}\end{center}
\end{frame}

\end{document}
